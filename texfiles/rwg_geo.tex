%!TEX root=../rwg.tex
\section{Analyzing Geo-distributed Datasets}
It is our belief that geo-distribution of cloud data will continue to
be a major difficulty for big data systems. While some prior work has
focused on solving particular problems that arise in storage across
multiple datacenters (e.g., COPS~\cite{Lloyd:2011:DSE:2043556.2043593}, or
Spanner~\cite{spanner-corbett}). However, we believe that the problem
of geo-distribution can be more generally reduced to scheduling
problems: how do we schedule resources and place data to satisfy
latency requirements and balance processing on less costly or loaded
data centers? Existing frameworks such as Iridium solve the wide-area
data analytics problem by casting it as a data placement optimization
problem~\cite{iridium-pu}.  Furthermore, even once this problem is
conceptualized as a set of constraints, how easy is it to solve these
constraints? Do we need approximate solutions, and how well will they
work?  In SWAG, rather than optimizing data placement, the framework
focuses on scheduling jobs to complete wide-area tasks and presents an
algorithm for scheduling based on greedy approximations~\cite{swag-hung}.

Secondly, we argue that the problem of geodistribution is really a
special case (multiple locations under single domain of control) of
the more general problem of data management and analysis across
multiple datacenters with multiple domains of control. When there are
multiple domains of control, additional constraints must be enforced
due to legal or organizational barriers which prevent data from
moving. This has been noted and discussed in the Geode
project~\cite{geode-vulimiri} which also presents an approach that
combines query planning with pseduo-distributed execution.  The same
basic questions arise in this situation as in more simply
geo-distributed systems, but there are also new ones, including concerns
about data privacy and security. When is it safe to share analysis
results or share processing? How can one domain trust the analysis of
another?
